\section{Порядок выполнения лабораторной работы №2}
\begin{enumerate}
\item Создать в папке с номером группы на рабочем столе папку \verb#Lab2# для файлов проекта.
\item Скопировать в созданную папку \verb#Lab2#, папку \verb#CMSIS# из папки \verb#LabWorkSTM32L/Materials# на рабочем столе.
\item Создать и настроить проект в среде разработки IAR. В качестве имени проекта указать \textit{lab2}, все файлы настроек проекта сохранить в папке \verb#Lab2# (Настройка проекта происходит, как указано в первой лабораторной работе).
\item Добавить в созданную папку с проектом файлы \verb\stm32l_discovery_lcd.c\, \verb\stm32l_discovery_lcd.h\, \verb\discover_board.h\ из папки \verb#LabWorkSTM32L/Materials/Discovery# и внести их в проект.
\item Добавить в созданную папку с проектом файл \verb\stm32l1xx_conf.h\ из папки \verb#LabWorkSTM32L/Materials#, добавить его в проект.
\item Добавить в созданную папку с проектом папку \verb#STM32L1xx_StdPeriph_Driver# из папки \verb#LabWorkSTM32L/Materials#, добавить в проект IAR файлы \verb\stm32l1xx_lcd.c\, \verb\stm32l1xx_lcd.h\, \verb\stm32l1xx_gpio.h\, \verb\stm32l1xx_gpio.с\, \verb\stm32l1xx_rcc.h\, \verb\stm32l1xx_rcc.c\.
\item В добавленные файлы из папки \verb#STM32L1xx_StdPeriph_Driver# добавить
\begin{verbatim}
#include "stm32l1xx_conf.h"
\end{verbatim}
\item Добавить в проект файл \verb\stm32l1xx.h\ из папки \verb#LabWorkSTM32L/Materials/CMSIS#
\verb#/DeviceSupport/ST/STM32L1xx#.
\item К добавленным папкам и файлам прописать пути во вкладке \textit{Preprocessor} в настройках проекта.
\item Получить от преподавателя 3 слова для вывода на ЖКИ.
\item Реализовать вывод первого заданного преподавателем слова используя регистры \textit{LCD->RAM[x]}.
\item Реализовать вывод второго заданного преподавателем слова используя библиотеку для работы с ЖКИ.
\item Реализовать бегущую строку на основе третьего полученного слова используя библиотеку для работы с ЖКИ.
\item Написать программу для микроконтроллера на языке Си.
\item Подключить отладочную плату STM32L-Discovery к компьютеру.
\item Загрузить программу в микроконтроллер и произвести ее отладку.
\item Отчет.
\end{enumerate}
\section{Пример выполнения лабораторной работы №2}
На ЖКИ выводится, с использованием регистров \textit{LCD->RAM[x]}, надпись \verb\KAF403\. С использованием библиотечной функции выводится бегущая строка \verb\MAI\, с использованием библиотечной функции выводится строка \verb\FRELA\.

Пример главной программы для микроконтроллера.
\begin{verbatim}
/*################## Секция include ##################*/
#include "stm32l1xx.h"
#include "stm32l_discovery_lcd.h"

/*################## Локальные переменные ##################*/
static uint32_t TimingDelay;

/*################## Прототипы функций ##################*/
void GPIO_Init(void);
void LCD_CONTR_Init(void);
void LAB2_Delay(uint32_t nTime);

/*################## Опредение функций ##################*/

/*
 * Функция инициализации портов микроконтроллера
 */ 
void GPIO_Init(void)
{
    /* Тактирование*/
    RCC->AHBENR |= (RCC_AHBENR_GPIOAEN | RCC_AHBENR_GPIOBEN | 
                    RCC_AHBENR_GPIOCEN);
    
    /* Режим работы, порты PA1 – PA3, PA8 – PA10, PA15 */
    GPIOA->MODER |= (GPIO_MODER_MODER1_1 | GPIO_MODER_MODER2_1 | 
                     GPIO_MODER_MODER3_1 | GPIO_MODER_MODER8_1 | 
                     GPIO_MODER_MODER9_1 | GPIO_MODER_MODER10_1 | 
                     GPIO_MODER_MODER15_1); 
    
    /* Режим работы, порты PB3 – PB5, PB8 – PB15 */
    GPIOB->MODER |= (GPIO_MODER_MODER3_1 | GPIO_MODER_MODER1_1 | 
                     GPIO_MODER_MODER5_1 | GPIO_MODER_MODER8_1 | 
                     GPIO_MODER_MODER9_1 | GPIO_MODER_MODER10_1 | 
                     GPIO_MODER_MODER11_1 | GPIO_MODER_MODER12_1 |
                     GPIO_MODER_MODER13_1 | GPIO_MODER_MODER14_1 | 
                     GPIO_MODER_MODER15_1); 
    
    /* Режим работы, порты PC0 – PC3, PC6 – PC11 */
    GPIOC->MODER |= (GPIO_MODER_MODER0_1 | GPIO_MODER_MODER1_1 | 
                     GPIO_MODER_MODER2_1 | GPIO_MODER_MODER3_1 |
                     GPIO_MODER_MODER6_1 | GPIO_MODER_MODER7_1 | 
                     GPIO_MODER_MODER8_1 | GPIO_MODER_MODER9_1 |  
                     GPIO_MODER_MODER10_1 | GPIO_MODER_MODER11_1); 
    
    /* Режим push – pull, порты PA1 – PA3, PA8 – PA10, PA15 */                 
    GPIOA->OTYPER &= ~(GPIO_OTYPER_OT_1 | GPIO_OTYPER_OT_2 |
                       GPIO_OTYPER_OT_3 | GPIO_OTYPER_OT_8 |
                       GPIO_OTYPER_OT_9 | GPIO_OTYPER_OT_10 |
                       GPIO_OTYPER_OT_15);
    
    /* Режим push – pull, порты PB3 – PB5, PB8 – PB15 */
    GPIOB->OTYPER &= ~(GPIO_OTYPER_OT_3 | GPIO_OTYPER_OT_4 | 
                       GPIO_OTYPER_OT_5 | GPIO_OTYPER_OT_8 | 
                       GPIO_OTYPER_OT_9 | GPIO_OTYPER_OT_10| 
                       GPIO_OTYPER_OT_11 | GPIO_OTYPER_OT_12 |
                       GPIO_OTYPER_OT_13 | GPIO_OTYPER_OT_14 | 
                       GPIO_OTYPER_OT_15);
    
    /* Режим push – pull, порты PC0 – PC3, PC6 – PC11 */
    GPIOC->OTYPER &= ~(GPIO_OTYPER_OT_0 | GPIO_OTYPER_OT_1 | 
                       GPIO_OTYPER_OT_2 | GPIO_OTYPER_OT_3 |
                       GPIO_OTYPER_OT_6 | GPIO_OTYPER_OT_7 | 
                       GPIO_OTYPER_OT_8 | GPIO_OTYPER_OT_9 |  
                       GPIO_OTYPER_OT_10 | GPIO_OTYPER_OT_11);
    
    /* Отключения подтягивающих резисторов, порты PA1 – PA3, 
       PA8 – PA10, PA15 */
    GPIOA->PUPDR &= ~(GPIO_PUPDR_PUPDR1 | GPIO_PUPDR_PUPDR2 |
                      GPIO_PUPDR_PUPDR3 | GPIO_PUPDR_PUPDR8 |
                      GPIO_PUPDR_PUPDR9 | GPIO_PUPDR_PUPDR10 |
                      GPIO_PUPDR_PUPDR15);

    /* Отключения подтягивающих резисторов, порты PB3 – PB5, PB8 – PB15 */
    GPIOB->PUPDR &= ~(GPIO_PUPDR_PUPDR3 | GPIO_PUPDR_PUPDR4 | 
                      GPIO_PUPDR_PUPDR5 | GPIO_PUPDR_PUPDR8 | 
                      GPIO_PUPDR_PUPDR9 | GPIO_PUPDR_PUPDR10| 
                      GPIO_PUPDR_PUPDR11 | GPIO_PUPDR_PUPDR12 |
                      GPIO_PUPDR_PUPDR13 | GPIO_PUPDR_PUPDR14 | 
                      GPIO_PUPDR_PUPDR15);

    /* Отключения подтягивающих резисторов, порты PC0 – PC3, PC6 – PC11 */
    GPIOC->PUPDR &= ~(GPIO_PUPDR_PUPDR0 | GPIO_PUPDR_PUPDR1 | 
                      GPIO_PUPDR_PUPDR2 | GPIO_PUPDR_PUPDR3 |
                      GPIO_PUPDR_PUPDR6 | GPIO_PUPDR_PUPDR7 | 
                      GPIO_PUPDR_PUPDR8 | GPIO_PUPDR_PUPDR9 |  
                      GPIO_PUPDR_PUPDR10 | GPIO_PUPDR_PUPDR11);
    
    /* Частота вывода информации, порты PA1 – PA3, PA8 – PA10, PA15 */
    GPIOA->OSPEEDR &= ~(GPIO_OSPEEDER_OSPEEDR1 | GPIO_OSPEEDER_OSPEEDR2 |
                        GPIO_OSPEEDER_OSPEEDR3 | GPIO_OSPEEDER_OSPEEDR8 |
                        GPIO_OSPEEDER_OSPEEDR9 | GPIO_OSPEEDER_OSPEEDR10 |
                        GPIO_OSPEEDER_OSPEEDR15);

    /* Частота вывода информации, порты PB3 – PB5, PB8 – PB15 */
    GPIOB->OSPEEDR &= ~(GPIO_OSPEEDER_OSPEEDR3 | GPIO_OSPEEDER_OSPEEDR4 | 
                        GPIO_OSPEEDER_OSPEEDR5 | GPIO_OSPEEDER_OSPEEDR8 | 
                        GPIO_OSPEEDER_OSPEEDR9 | GPIO_OSPEEDER_OSPEEDR10| 
                        GPIO_OSPEEDER_OSPEEDR11 | GPIO_OSPEEDER_OSPEEDR12 |
                        GPIO_OSPEEDER_OSPEEDR13 | GPIO_OSPEEDER_OSPEEDR14 | 
                        GPIO_OSPEEDER_OSPEEDR15);

    /* Частота вывода информации, порты PC0 – PC3, PC6 – PC11 */
    GPIOC->OSPEEDR &= ~(GPIO_OSPEEDER_OSPEEDR0 | GPIO_OSPEEDER_OSPEEDR1 | 
                        GPIO_OSPEEDER_OSPEEDR2 | GPIO_OSPEEDER_OSPEEDR3 |
                        GPIO_OSPEEDER_OSPEEDR6 | GPIO_OSPEEDER_OSPEEDR7 | 
                        GPIO_OSPEEDER_OSPEEDR8 | GPIO_OSPEEDER_OSPEEDR9 |  
                        GPIO_OSPEEDER_OSPEEDR10 | GPIO_OSPEEDER_OSPEEDR11);

    /* Установка альтернативной функции */
    GPIOA->AFR[0] |= 0xBBB0;
    GPIOA->AFR[1] |= 0xB0000BBB;

    GPIOB->AFR[0] |= 0xBBB000;
    GPIOB->AFR[1] |= 0xBBBBBBBB;

    GPIOC->AFR[0] |= 0xBB00BBBB;
    GPIOC->AFR[1] |= 0xBBBB;
}

/*
 * Функция инициализации контроллера ЖКИ
 */
void LCD_CONTR_Init(void)
{
    /* Тактирование */
    RCC->APB1ENR |= RCC_APB1ENR_PWREN|RCC_APB1ENR_LCDEN;

    /* Снимается защита от записи в регистр RCC_CSR */
    PWR->CR |= PWR_CR_DBP;

    /* Сброс источника тактирования */
    RCC->CSR |= RCC_CSR_RTCRST;
    RCC->CSR &= ~RCC_CSR_RTCRST;

    /* Выбираем LSE (low-speed external) — 
       низкочастотный внешний источник */
    RCC->CSR |= RCC_CSR_LSEON;

    /* Ждем стабилизацию генератора */
    while(!(RCC->CSR&RCC_CSR_LSERDY));

    /* Выбор внешнего НЧ генератора в качестве источника 
       тактового сигнала */
    RCC->CSR |= RCC_CSR_RTCSEL_LSE;

    /* Установка bias = 1/3 */
    LCD->CR &= ~LCD_CR_BIAS;
    LCD->CR |= LCD_CR_BIAS_1;

    /* Установка duty = 1/3 */
    LCD->CR &= ~LCD_CR_DUTY;
    LCD->CR |= LCD_CR_DUTY_0 | LCD_CR_DUTY_1;

    /* Используем ф-цию переназначения выводов */
    LCD->CR |= LCD_CR_MUX_SEG;

    /* Устанавливаем коэффициенты деления частоты тактового сигнала */
    LCD->FCR &= ~LCD_FCR_PS;
    LCD->FCR &= ~LCD_FCR_DIV;

    /* ck_ps = LCDCLK / 16, ck_div = ck_ps / 17 */
    LCD->FCR |= 0x1040000;

    /* Установка контрасности */
    LCD->FCR &= ~LCD_FCR_CC;
    LCD->FCR |= LCD_FCR_CC_1;

    /* Ждем синхронизацию регистра LCD_FCR */
    while(!(LCD->SR&LCD_SR_FCRSR));

    /* Внутренний step-up converter для формирования V_lcd */
    LCD->CR &= ~LCD_CR_VSEL;

    /* Разрешение работы ЖКИ контроллера */
    LCD->CR |= LCD_CR_LCDEN;

    /* Ждем готовность step-up converter */
    while(!(LCD->SR&LCD_SR_RDY));

    /* Ждем готовность ЖКИ контроллера */
    while(!(LCD->SR&LCD_SR_ENS));
}

/*
 * Функция временной задержки
 * Входные данные:
 *  nTime - время задержки [мс]
 */
void LAB2_Delay(uint32_t nTime)
{
    TimingDelay = nTime;
    while(TimingDelay != 0);
}

/*
 * Обработчик прерывания системного таймера
 */
void SysTick_Handler(void)
{
    if (TimingDelay != 0x00)
    {
        TimingDelay--;
    }
}

/*################## Программа ##################*/
void main(void)
{
    GPIO_Init();
    LCD_CONTR_Init();
    SysTick_Config(2097);   /* Конфигурирование системного таймера 
                               SysTick, определен в core_cm3.h */
    while(1)
    {
        while(LCD->SR & LCD_SR_UDR);
        
        /* Вывод информации на ЖКИ через регистры */
        LCD->RAM[0] = 0x2E369985;
        LCD->RAM[2] = 0x2F2EF880;
        LCD->RAM[4] = 0x10000000;
        LCD->RAM[6] = 0x1;

        /* Устанавливаем Update display request */
        LCD->SR |= LCD_SR_UDR;
        
        LAB2_Delay(3000);
        LCD_GLASS_Clear();

        /* Бегущая строка, 1-количество повторений, 200 - задержка в
           миллисекундах перед обновлением информации на ЖКИ */
        LCD_GLASS_ScrollSentence(" MAI", 1, 200);

        LCD_GLASS_DisplayString("FRELA");
        LAB2_Delay(2000);
    }
}
\end{verbatim}